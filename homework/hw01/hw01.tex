%----------------------------------------------------------------------------------------
%	PACKAGES AND OTHER DOCUMENT CONFIGURATIONS
%----------------------------------------------------------------------------------------

\documentclass{article}

\usepackage{fancyhdr} % Required for custom headers
\usepackage{lastpage} % Required to determine the last page for the footer
\usepackage{extramarks} % Required for headers and footers
\usepackage{graphicx} % Required to insert images
\usepackage{listings}
\usepackage{color}
\usepackage{xcolor}
\usepackage{caption}
\usepackage{enumitem}
\usepackage{amsmath}
\usepackage{tikz}
\usetikzlibrary{calc,shapes.multipart,chains,arrows}

\DeclareCaptionFont{white}{\color{white}}
\DeclareCaptionFormat{listing}{%
\parbox{\textwidth}{\colorbox{gray}{\parbox{\textwidth}{#1#2#3}}\vskip-4pt}}
\captionsetup[lstlisting]{format=listing,labelfont=white,textfont=white}

% Margins
\topmargin=-0.45in
\evensidemargin=0in
\oddsidemargin=0in
\textwidth=6.5in
\textheight=9.0in
\headsep=0.25in 

\linespread{1.1} % Line spacing

% Set up the header and footer
\pagestyle{fancy}
\lhead{\hmwkAuthorName} % Top left header
\chead{\hmwkClass\ (\hmwkClassInstructor\ \hmwkClassTime): \hmwkTitle} % Top center header
\rhead{\hmwkDueDate} % Top right header
\lfoot{\lastxmark} % Bottom left footer
\cfoot{} % Bottom center footer
\rfoot{Page\ \thepage\ of\ \pageref{LastPage}} % Bottom right footer
\renewcommand\headrulewidth{0.4pt} % Size of the header rule
\renewcommand\footrulewidth{0.4pt} % Size of the footer rule

\setlength\parindent{0pt} % Removes all indentation from paragraphs

%----------------------------------------------------------------------------------------
%	DOCUMENT STRUCTURE COMMANDS
%	Skip this unless you know what you're doing
%----------------------------------------------------------------------------------------

\setcounter{secnumdepth}{0} % Removes default section numbers
\newcounter{homeworkProblemCounter} % Creates a counter to keep track of the number of problems

\newcommand{\homeworkProblemName}{}
\newenvironment{homeworkProblem}[1][Problem \arabic{homeworkProblemCounter}]{ % Makes a new environment called homeworkProblem which takes 1 argument (custom name) but the default is "Problem #"
\stepcounter{homeworkProblemCounter} % Increase counter for number of problems
\renewcommand{\homeworkProblemName}{#1} % Assign \homeworkProblemName the name of the problem
\section{\homeworkProblemName} % Make a section in the document with the custom problem count
}

%----------------------------------------------------------------------------------------
%   COLORS AND LANGUAGAGE
%----------------------------------------------------------------------------------------

\lstset{
    frame=lrb,xleftmargin=\fboxsep,xrightmargin=-\fboxsep,language=Java,basicstyle=\ttfamily,
    breaklines=true,columns=fullflexible,keepspaces=true,escapeinside={\%*}{*)}
       }

%----------------------------------------------------------------------------------------
%	NAME AND CLASS SECTION
%----------------------------------------------------------------------------------------

\newcommand{\hmwkTitle}{Homework\ \#1} % Assignment title
\newcommand{\hmwkDueDate}{ Thursday,\ September\ 10,\ 2015} % Due date
\newcommand{\hmwkClass}{COT\ 5310} % Course/class
\newcommand{\hmwkClassTime}{6:15pm} % Class/lecture time
\newcommand{\hmwkClassInstructor}{Bobadilla} % Teacher/lecturer
\newcommand{\hmwkAuthorName}{Musa V. Ahmed} % Your name

%----------------------------------------------------------------------------------------

\begin{document}
\belowcaptionskip=-10pt

%----------------------------------------------------------------------------------------
%	PROBLEM 1
%----------------------------------------------------------------------------------------

\begin{homeworkProblem}
    Solve Sipser exercises 0.3, 0.4, 0.5, and problem 0.10.

    \vspace{5 mm}
    0.3 
    \begin{enumerate}[label=\alph*.]
        \item No.
        \item Yes.
        \item ${x, y, z}$
        \item ${x, y}$
        \item ${xx, xy, yx, yy, zx, zy, xz, yz}$
        \item ${{x,y}, {x}, {y}}$
    \end{enumerate}

    0.4 The cartesian product has $a*b$ elements. Since the cartesian product is $b$ ordered 
    pairs for each of the $m$ elements.

    0.5 There are $2^c$ elements in the power set of $C$. Since the subsets in the power set 
    either include an element of $C$ or they do not then it holds that there are $2^|C|$ 
    elements in the power set.

    0.10 In the step where we divide both sides by $(a-b)$ if $a=b$ then $(a-b)=0$. This leads 
    to a divide by zero error.
\end{homeworkProblem}
\clearpage

%----------------------------------------------------------------------------------------

%----------------------------------------------------------------------------------------
%	PROBLEM 2
%----------------------------------------------------------------------------------------

\begin{homeworkProblem}
    For a DFA, $M=(Q,\sum, \delta, q_0, F)$ in which the set of states is $Q=\{q_1, q_2, q_3, q_4, q_5\}$, 
    $\sum = \{a, b\}$, $q_0=q_1$, $F=\{q_2, q_3, q_4, q_5\}$, and $\delta$ is specified by the table: \\

    \begin{center}
        \begin{tabular}{l | l | l | l | l | l}
            \hline
            $\delta$ & $q_1$ & $q_2$ & $q_3$ & $q_4$ & $q_5$ \\ \hline
            a & $q_1$ & $q_3$ & $q_5$ & $q_2$ & $q_4$ \\ \hline
            b & $q_2$ & $q_4$ & $q_1$ & $q_3$ & $q_5$ \\ \hline
        \end{tabular}
    \end{center}

    Do the following:
    \begin{enumerate}[label=(\alph*)]
        \item Draw the state diagram of the DFA.
        \item For the strings below, give the corresponding computation of the automaton 
            and say whether it accepts or rejects them. The definition of computation 
            is given in page 40.
            \begin{enumerate}
                \item baab
                \item abbb
                \item bbba
            \end{enumerate}
        \item Give a succinct English description of the string accepted by $M$.
    \end{enumerate}


\end{homeworkProblem}
\clearpage

%----------------------------------------------------------------------------------------

%----------------------------------------------------------------------------------------
%	PROBLEM 3
%----------------------------------------------------------------------------------------

\begin{homeworkProblem}
    For each of the following languages give a state diagram of a DFA that recognize it. 
    The alphabet is $\sum=\{0, 1\}$

    \begin{enumerate}[label=(\alph*)]
        \item \{$w|w$ does not contain 000 or 11 as a substring\}
        \item \{$w|w$ contain at least two 0's and at least two 1's\}. The 0's and 1's do not 
            need to be consecutive.
    \end{enumerate}


\end{homeworkProblem}
\clearpage

%----------------------------------------------------------------------------------------

%----------------------------------------------------------------------------------------
%	PROBLEM 4
%----------------------------------------------------------------------------------------

\begin{homeworkProblem}
    Solve Sipser exercises 1.6b, 1.6d, 1.5c, 1.4c.


\end{homeworkProblem}
\clearpage

%----------------------------------------------------------------------------------------

\end{document}
