%----------------------------------------------------------------------------------------
%	PACKAGES AND OTHER DOCUMENT CONFIGURATIONS
%----------------------------------------------------------------------------------------

\documentclass{article}

\usepackage{fancyhdr} % Required for custom headers
\usepackage{lastpage} % Required to determine the last page for the footer
\usepackage{extramarks} % Required for headers and footers
\usepackage{graphicx} % Required to insert images
\usepackage{listings}
\usepackage{color}
\usepackage{xcolor}
\usepackage{caption}
\usepackage{enumitem}
\usepackage{amsmath}
\usepackage{tikz}
\usetikzlibrary{calc,shapes.multipart,chains,arrows}

\DeclareCaptionFont{white}{\color{white}}
\DeclareCaptionFormat{listing}{%
\parbox{\textwidth}{\colorbox{gray}{\parbox{\textwidth}{#1#2#3}}\vskip-4pt}}
\captionsetup[lstlisting]{format=listing,labelfont=white,textfont=white}

% Margins
\topmargin=-0.45in
\evensidemargin=0in
\oddsidemargin=0in
\textwidth=6.5in
\textheight=9.0in
\headsep=0.25in 

\linespread{1.1} % Line spacing

% Set up the header and footer
\pagestyle{fancy}
\lhead{\hmwkAuthorName} % Top left header
\chead{\hmwkClass\ (\hmwkClassInstructor\ \hmwkClassTime): \hmwkTitle} % Top center header
\rhead{\hmwkDueDate} % Top right header
\lfoot{\lastxmark} % Bottom left footer
\cfoot{} % Bottom center footer
\rfoot{Page\ \thepage\ of\ \pageref{LastPage}} % Bottom right footer
\renewcommand\headrulewidth{0.4pt} % Size of the header rule
\renewcommand\footrulewidth{0.4pt} % Size of the footer rule

\setlength\parindent{0pt} % Removes all indentation from paragraphs

%----------------------------------------------------------------------------------------
%	DOCUMENT STRUCTURE COMMANDS
%	Skip this unless you know what you're doing
%----------------------------------------------------------------------------------------

\setcounter{secnumdepth}{0} % Removes default section numbers
\newcounter{homeworkProblemCounter} % Creates a counter to keep track of the number of problems

\newcommand{\homeworkProblemName}{}
\newenvironment{homeworkProblem}[1][Problem \arabic{homeworkProblemCounter}]{ % Makes a new environment called homeworkProblem which takes 1 argument (custom name) but the default is "Problem #"
\stepcounter{homeworkProblemCounter} % Increase counter for number of problems
\renewcommand{\homeworkProblemName}{#1} % Assign \homeworkProblemName the name of the problem
\section{\homeworkProblemName} % Make a section in the document with the custom problem count
}

%----------------------------------------------------------------------------------------
%   COLORS AND LANGUAGAGE
%----------------------------------------------------------------------------------------

\lstset{
    frame=lrb,xleftmargin=\fboxsep,xrightmargin=-\fboxsep,language=Java,basicstyle=\ttfamily,
    breaklines=true,columns=fullflexible,keepspaces=true,escapeinside={\%*}{*)}
       }

%----------------------------------------------------------------------------------------
%	NAME AND CLASS SECTION
%----------------------------------------------------------------------------------------

\newcommand{\hmwkTitle}{Homework\ \#2} % Assignment title
\newcommand{\hmwkDueDate}{ Thursday,\ September\ 22,\ 2015} % Due date
\newcommand{\hmwkClass}{COT\ 5310} % Course/class
\newcommand{\hmwkClassTime}{6:15pm} % Class/lecture time
\newcommand{\hmwkClassInstructor}{Bobadilla} % Teacher/lecturer
\newcommand{\hmwkAuthorName}{Musa V. Ahmed} % Your name

%----------------------------------------------------------------------------------------

\begin{document}
\belowcaptionskip=-10pt

%----------------------------------------------------------------------------------------
%	PROBLEM 1
%----------------------------------------------------------------------------------------

\begin{homeworkProblem}
    Solve Sipser exercises (2nd edition) 1.7b, 1.7e, 1.16, 1.19, 1.21.


\end{homeworkProblem}
\clearpage

%----------------------------------------------------------------------------------------

%----------------------------------------------------------------------------------------
%	PROBLEM 2
%----------------------------------------------------------------------------------------

\begin{homeworkProblem}
    Find an equivalent NFA for the following regular expression:\\ \\
    $R=(0(10)^* \cup (1(0 \cup 1)^*)$


\end{homeworkProblem}
\clearpage

%----------------------------------------------------------------------------------------

%----------------------------------------------------------------------------------------
%	PROBLEM 3
%----------------------------------------------------------------------------------------

\begin{homeworkProblem}
    For the following languages, give a corresponding regular expression. The
    languages are defined over the alphabet $\Sigma=\{a,b\}$    

    \begin{enumerate}[label=\alph*.]
        \item $A_1$: The set of all strings that contain "a" as a substring
        \item $A_2$: The set of all strings that do not contain "bb" as a
            substring
        \item $A_3$: The set of all string whose length is exactly three.
    \end{enumerate}


\end{homeworkProblem}
\clearpage

%----------------------------------------------------------------------------------------

%----------------------------------------------------------------------------------------
%	PROBLEM 4
%----------------------------------------------------------------------------------------

\begin{homeworkProblem}
    Using the pumping lemma, prove that:\\ \\

    $A_4 = \{w \in \Sigma^* |\  w\ contains\ more\ a's\ than\ b's\}$\\
    with $\Sigma=\{a,b\}$, is not a regular language.


\end{homeworkProblem}
\clearpage

%----------------------------------------------------------------------------------------

%----------------------------------------------------------------------------------------
%	PROBLEM 5
%----------------------------------------------------------------------------------------

\begin{homeworkProblem}
    Are the following languages regular? Prove your answers.

    \begin{itemize}
        \item $C_1 = \{a^pb^qa^{p=q} \in \Sigma^* |\ p \geq 0, q \geq 0\}$
        \item $C_2 = \{a^{\dbinom{n}{s}} \in \Sigma^* |\ n \geq 2\}$
    \end{itemize}
\end{homeworkProblem}
\clearpage

%----------------------------------------------------------------------------------------

\end{document}
